% !TeX program = pdflatex
\documentclass[11pt,a4paper]{article}
\usepackage[margin=1in]{geometry}
\usepackage{graphicx}
\usepackage{booktabs}
\usepackage{hyperref}
\usepackage{amsmath}
\usepackage{siunitx}
\usepackage{enumitem}
\usepackage{xcolor}

% Metadata
\title{The Plastic Tide: Forecasting Macroplastic Pollution in Major River Systems}
\author{Team Name (Group XXX)\\ Member 1 \and Member 2 \and Member 3 \and Member 4}
\date{Submission Date: 30 November 2025}

\begin{document}

% Front Page
\begin{titlepage}
    \centering
    {\Huge \bfseries The Plastic Tide: Forecasting Macroplastic Pollution in Major River Systems\\[1.5ex]}
    \vspace{1cm}
    {\Large Data Science Project Report}\\[1cm]
    {\large Group ID: XXX}\\[0.5cm]

    {\large Team Members:}\\
    {Member 1 (ID)\\ Member 2 (ID)\\ Member 3 (ID)\\ Member 4 (ID)}\\[1cm]

    {\large Date of Submission: 30.11.2025, 18:00}\\[2cm]

    \vfill
    {\large Declaration: This report is our original work.}
\end{titlepage}

\tableofcontents
\newpage

\section{Understanding the Theory}
\label{sec:theory}
This section outlines the theory and methods to model riverine plastic waste outflow as a supervised regression problem. We consider multicollinearity and heteroscedasticity. We compare linear regularized models (Ridge, Lasso) and tree ensembles (Gradient Boosting, XGBoost). We discuss feature engineering (population within buffers, GDP per capita, waste management indices), scaling, train/validation/test splits, cross-validation, and evaluation metrics (R\textsuperscript{2}, MAPE).

\subsection{Problem Formulation}
Let $y$ denote annual plastic outflow (kg/year) for a river catchment, with features $\mathbf{x}$ derived from socioeconomic and geographic data. We fit $\hat{y} = f(\mathbf{x})$ minimizing a loss function appropriate for regression.

\subsection{Regularized Linear Models}
We use Ridge (L2) and Lasso (L1) regression to address multicollinearity and perform feature selection. Hyperparameters are tuned via cross-validation.

\subsection{Gradient Boosting Models}
Tree-based ensembles like XGBoost capture nonlinearities and interactions, and provide feature importance for interpretability.

\section{Implementation}
\label{sec:implementation}
We implement the pipeline in a Python Jupyter Notebook (\texttt{notebooks/plastic_tide.ipynb}): data loading, cleaning, feature engineering, model training (Ridge, Lasso, XGBoost), hyperparameter tuning, and evaluation. We log metrics and produce plots (parity plots, residuals, feature importances).

\section{Experimental Results}
\label{sec:results}
We report R\textsuperscript{2} and MAPE on a held-out test set, along with ablation analyses. Include tables and figures generated by the notebook.

\section{Conclusion and Policy Implications}
Summarize findings and discuss which features most strongly predict riverine plastic outflow and the implications for targeted interventions.

\section*{Individual Contributions}
\begin{itemize}[leftmargin=*]
  \item Member 1: Data acquisition and cleaning
  \item Member 2: Feature engineering and EDA
  \item Member 3: Modeling and validation
  \item Member 4: Report writing and visualization
\end{itemize}

\section*{References}
\begin{itemize}[leftmargin=*]
  \item Meijer et al. (2021), ``More than 1000 rivers account for 80\% of global riverine plastic emissions.''
  \item World Bank Open Data: \url{https://data.worldbank.org/}
  \item GRID3 Population Density: \url{https://grid3.org/}
\end{itemize}

\end{document}
